\input ../macros.tex

\beginsection Matrix Calculus

\item{A.}\emph{Multivariable Calculus}
\itemitem{1.} Scope: multi-valued functions of multiple variables\medskip

\itemitem{}\vbox{\offinterlineskip
% ===== Preamble =====
\halign{
\vrule                % Vertical line
height2.75ex          % Add extra whitespace on top of rows
depth1.25ex           % Add extra whitespace at bottom of rows
width0.6pt#&          % Make the line a little thicker compared to normal
\enskip #\hfil\enskip\vrule&                       % Column 1
\enskip\hfil #\hfil\enskip\vrule&                  % Column 2
\enskip\hfil #\hfil\enskip\vrule&                  % Column 3
\enskip\hfil #\hfil\enskip&#\vrule width 0.6pt\cr  % Column 4
\noalign{\hrule height 0.6pt}
% ===== Row 1 =====
&&\bf Scalar&\bf Vector&\bf Matrix&\crl
% ===== Row 2 =====
&\bf Dependent var&$y\in\R$&${\bf y}\in\R^m$&$Y\in\R^{m\times n}$&\crl
% ===== Row 3 =====
&\bf Independent var&$x\in\R$&${\bf x}\in\R^n$&$X\in\R^{p\times q}$&\cr
\noalign{\hrule height 0.6pt}
}}

\itemitem{2.} Notations
\itemitemitem{(a)}\emph{Matrix calculus} collects the partial derivatives
${\partial y_i\over\partial x_j}$ into a matrix; two variants:\smallskip

\itemitemitem{}\vbox{\offinterlineskip
% ===== Preamble =====
\halign{
\vrule                % Vertical line
height2.75ex          % Add extra whitespace on top of rows
depth1.25ex           % Add extra whitespace at bottom of rows
width0.6pt#&          % Make the line a little thicker compared to normal
\enskip #\hfil\enskip\vrule&                       % Column 1
\enskip\hfil #\hfil\enskip\vrule&                  % Column 2
\enskip\hfil #\hfil\enskip&#\vrule width 0.6pt\cr  % Column 3
\noalign{\hrule height 0.6pt}
% ===== Row 1 =====
&&\low{2}{\bf Indep var}&\low{2}{\bf Dep var}&\cr
&&\up{2}{\bf Components}&\up{2}{\bf Components}&\crl
% ===== Row 2 =====
&\low{2}{\bf Numerator layout}&\low{9}{Columns}&\low{9}{Rows}&\cr
\omit&\omit&\omit&\cr
&\up{2}{\bf /Jacobian formulation}&&&\crl
% ===== Row 3 =====
&\low{2}{\bf Denominator layout}&\low{9}{Rows}&\low{9}{Columns}&\cr
\omit&\omit&\omit&\cr
&\up{2}{\bf /Hessian formulation}&&&\cr
\noalign{\hrule height 0.6pt}
}}

\itemitemitem{(b)} Physics favors \emph{tensor index notation} (Ricci calculus or
Einstein summation)\smallskip
% ; two variants:
% \itemitemitemitem{II.} Einstein summation
\item{B.}\emph{Matrix derivatives} (using numerator layout)\footnote{${}^1$}{
Tables copied from the Wikipedia article on ``Matrix calculus''.}
\itemitem{1.}Scalar- or vector-values variables\medskip

\itemitem{}\vbox{\offinterlineskip
% ===== Preamble =====
\halign{
\vrule                % Vertical line
height2.75ex          % Add extra whitespace on top of rows
depth1.25ex           % Add extra whitespace at bottom of rows
width0.6pt#&          % Make the line a little thicker compared to normal
\enskip\hfil #\hfil\enskip\vrule&                  % Column 1
\enskip\hfil #\hfil\enskip\vrule&                  % Column 2
\enskip\hfil #\hfil\enskip\vrule&                  % Column 3
\enskip\hfil #\hfil\enskip\vrule&                  % Column 4
\enskip\hfil #\hfil\enskip\vrule&                  % Column 5
\enskip\hfil #\hfil\enskip\vrule&                  % Column 6
\enskip\hfil #\hfil\enskip\vrule&                  % Column 7
\enskip\hfil #\hfil\enskip&#\vrule width 0.6pt\cr  % Column 8
\noalign{\hrule height 0.6pt}
% ===== Row 1 =====
&\multirow{2}{}&
\multirow{3}{\bf Scalar $y$ ($1\times 1$)}&
\multirow{3}{\bf Vector y ($m\times 1$)}&\cr
\omit&\omit&\omit&\omit\hrulefill&\omit\hrulefill&\omit\hrulefill&\omit\hrulefill&
\omit\hrulefill&\omit\hrulefill&\cr
% ===== Row 2 =====
&\multispan2\hfil\vrule&\bf Notation&
\bf Dim&
\bf Name&
\bf Notation&
\bf Dim&
\bf Name&\crl
% ===== Row 3 =====
&\low{2}{\bf Scalar $x$}&
{\bf Num}&
\low{17}{$\frac{\partial y}{\partial x}=\frac{dy}{dx}$}&
\low{9}{$1\times 1$}&
\low{2}{Scalar}&
\low{17}{$\frac{\partial{\bf y}}{\partial x}$}&
$m\times 1$&
\low{2}{Tangent}&\cr
\omit&\omit&\omit\hrulefill&\omit&\omit&\omit&\omit&\omit\hrulefill&\omit&\cr
% ===== Row 4 =====
&\up{2}{($1\times 1$)}&
{\bf Denom}&&&
\up{2}{Derivative}&&
$1\times m$&
\up{2}{Vector}&\cr
\noalign{\hrule height 0.6pt}
% ===== Row 5 =====
&\low{2}{\bf Vector x}&
{\bf Num}&
\low{17}{$\frac{\partial y}{\partial\bf x}=\nabla y$}&
$1\times n$&
\low{2}{Gradient}&
\low{17}{$\frac{\partial\bf y}{\partial\bf x}=J$}&
$1\times n$&
\low{9}{Jacobian}&\cr
\omit&\omit&\omit\hrulefill&\omit&\omit\hrulefill&\omit&\omit&\omit\hrulefill&\omit&\cr
% ===== Row 6 =====
&\up{2}{($n\times 1$)}&
{\bf Denom}&&
$n\times 1$&
\up{2}{Vector}&&
$n\times m$&&\cr
\noalign{\hrule height 0.6pt}
}}

\itemitemitem{(a)}\emph{Tangent vector} ${\partial{\bf y}\over\partial x}=
\left[{\partial y_1\over\partial x}\enskip\cdots\enskip{\partial y_m\over\partial x}
\right]^T\in\R^{m\times 1}$ 
\itemitemitem{(b)}\emph{Gradient vector} $\nabla y={\partial y\over\partial{\bf x}}
=\left[{\partial y\over\partial x_1}\enskip\cdots\enskip{\partial y\over\partial x_n}
\right]\in\R^{1\times n}$ 
\itemitemitem{(c)}\emph{Jacobian} $J={\partial{\bf y}\over\partial{\bf x}}=
\left({\partial y_i\over\partial x_j}\right)_{ij}\in\R^{m\times n}$ 

\itemitem{2.} Scalar- and matrix-valued variables\medskip

\itemitem{}\vbox{\offinterlineskip
% ===== Preamble =====
\halign{
\vrule                % Vertical line
height2.75ex          % Add extra whitespace on top of rows
depth1.25ex           % Add extra whitespace at bottom of rows
width0.6pt#&          % Make the line a little thicker compared to normal
\enskip\hfil #\hfil\enskip\vrule&                  % Column 1
\enskip\hfil #\hfil\enskip\vrule&                  % Column 2
\enskip\hfil #\hfil\enskip\vrule&                  % Column 3
\enskip\hfil #\hfil\enskip\vrule&                  % Column 4
\enskip\hfil #\hfil\enskip\vrule&                  % Column 5
\enskip\hfil #\hfil\enskip\vrule&                  % Column 6
\enskip\hfil #\hfil\enskip\vrule&                  % Column 7
\enskip\hfil #\hfil\enskip&#\vrule width 0.6pt\cr  % Column 8
\noalign{\hrule height 0.6pt}
% ===== Row 1 =====
&\multirow{2}{}&
\multirow{3}{\bf Scalar $y$ ($1\times 1$)}&
\multirow{3}{{\bf Matrix} $Y$ ($m\times n$)}&\cr
\omit&\omit&\omit&\omit\hrulefill&\omit\hrulefill&\omit\hrulefill&\omit\hrulefill&
\omit\hrulefill&\omit\hrulefill&\cr
% ===== Row 2 =====
&\multispan2\hfil\vrule&\bf Notation&
\bf Dim&
\bf Name&
\bf Notation&
\bf Dim&
\bf Name&\crl
% ===== Row 3 =====
&\low{2}{\bf Scalar $x$}&
{\bf Num}&&&&
\low{17}{$\frac{\partial Y}{\partial x}$}&
$m\times n$&
\low{2}{Tangent}&\cr
\omit&\omit&\omit\hrulefill&\omit&\omit&\omit&\omit&\omit\hrulefill&\omit&\cr
% ===== Row 4 =====
&\up{2}{($1\times 1$)}&
{\bf Denom}&&&&&
$n\times m$&
\up{2}{Matrix}&\cr
\noalign{\hrule height 0.6pt}
% ===== Row 5 =====
&\low{2}{{\bf Matrix} $X$}&
{\bf Num}&
\low{17}{$\frac{\partial y}{\partial X}$}&
$q\times p$&
\low{2}{Gradient}&&&&\cr
\omit&\omit&\omit\hrulefill&\omit&\omit\hrulefill&\omit&\omit&\omit&\omit&\cr
% ===== Row 6 =====
&\up{2}{($p\times q$)}&
{\bf Denom}&&
$p\times q$&
\up{2}{Matrix}&&&&\cr
\noalign{\hrule height 0.6pt}
}}

\smallskip\itemitemitem{(a)}\emph{Tangent matrix} ${\partial Y\over\partial x}=
\left({\partial y_{ij}\over\partial x}\right)_{ij}\in\R^{m\times n}$

\itemitemitem{(b)}\emph{Gradient matrix} ${\partial y\over\partial X}=
\left({\partial x_{ij}\over\partial y}\right)_{ji}\in\R^{q\times p}$

\bye
