\input epsf
\input ../macros.tex

\beginsection Dot products (or: norms and angles)

\item{A.}\emph{Euclidean norm} (/magnitude/length) $\|x\|\in\R$ of $x\in\R^m$:
$(x^Tx)^{1/2}$, generalizing $\|x\|=\sqrt{x^2}$\smallskip
\itemitem{1.}Properties:\smallskip
\itemitemitem{(a)}\emph{Positive definiteness}: $\|x\|>0$ (non-negativity) and
$\|x\|=0\leftrightarrow x=0$ (definiteness)\smallskip
\itemitemitem{(b)}\emph{Nonnegative homogeneity}: $\|\beta x\|=|\beta\|\|x\|$
\smallskip
\itemitemitem{(c)}\emph{Triangle ineq/sub-additivity}: $\|x+y\|\leq\|x\|+\|y\|$
\smallskip
\itemitem{2.}\emph{Unit vector}: any $a\in\R^m$ s.t. $\|a\|=1$\smallskip
\itemitemitem{(a)}\emph{Standard unit vectors}: $e_1,\ldots,e_m\in\R^m$ where
$e_i(j)=1$ iff $i=j$ and $0$ otherwise\smallskip
\itemitem{3.}\emph{Direction} of $a\in\R^m$: the unit vector $a/\|a\|\in\R^m$
(so vector = magnitude $\times$ direction)\smallskip

\item{B.}\emph{Cauchy-Schwarz inequality}: $|a^Tb|\leq\|a\|\|b\|$\smallskip
\itemitem{1.} Proves triangle ineq: $\|a+b\|^2=\|a\|^2+2a^Tb+\|b\|^2\leq\|a\|^2
+2\|a\|\|b\|+\|b\|^2=(\|a\|+\|b\|)^2$\smallskip
\itemitem{2.}\emph{Proof}. Immediate if $a=0$ or $b=0$. Else, for $\alpha=
\|a\|$, $\beta=\|b\|$,\smallskip
\itemitemitem{(a)} $0\leq\|\beta a-\alpha b\|^2=\beta^2\|a\|^2-2\alpha\beta
(a^Tb)+\alpha^2\|b\|^2=2\|a\|^2\|b\|^2-2\|a\|\|b\|(a^Tb)$\smallskip
\itemitemitem{(b)} Division by $2\|a\|\|b\|$ gives $0\leq\|a\|\|b\|-a^Tb$, iff
$a^Tb\leq\|a\|\|b\|$\smallskip
\itemitemitem{(c)} By (b), $-a^Tb=(-a)^Tb\leq\|-a\|\|b\|=\|a\|\|b\|$\smallskip

\item{C.}(Smallest positive) \emph{angle} $\Theta=\angle(a,b)$ between non-zero
$a,b\in\R^m$: $\arccos\left({a^Tb\over\|a\|\|b\|}\right)\in[0,\pi]$\smallskip
\itemitem{1.}\emph{Well-definedness}: ${a^Tb\over\|a\|\|b\|}\in
{\rm Dom}_{\arccos}=[-1,1]$ by Cauchy-Schwarz\smallskip
\itemitem{2.}\emph{Proof} Letting $c=a-b$,\smallskip
\itemitemitem{(a)} $\|c\|^2=\|a\|^2+\|b\|^2-2\|a\|\|b\|\cos\Theta$ by the law
of cosines\smallskip
\itemitemitem{(b)} At the same time $\|c\|^2=\sum_{i=1}^m(a_i^2-2a_ib_i+b_i^2)$
\smallskip
\itemitemitem{(c)} By (a) and (b), $2\|a\|\|b\|\cos\Theta=2a^Tb$, so $\cos\Theta
={a^Tb\over\|a\|\|b\|}$\smallskip
\itemitem{3.}\emph{Properties}: $\angle(a,b)=\angle(b,a)$ and $\angle(\alpha a,
\alpha b)=\angle(a,b)$ ($\alpha\not=0$)\smallskip
\itemitem{4.} Acute and obtuse angles:\smallskip
\itemitem{}\vbox{\offinterlineskip
% ===== Preamble =====
\halign{
\vrule                % Vertical line
height2.75ex          % Add extra whitespace on top of rows
depth1.25ex           % Add extra whitespace at bottom of rows
width0.6pt#&          % Make the line a little thicker compared to normal
\enskip #\hfil\enskip\vrule&                  % Column 1
\enskip #\hfil\enskip\vrule&                  % Column 2
\enskip #\hfil\enskip\vrule&                  % Column 3
\enskip #\hfil\enskip&#\vrule width 0.6pt\cr  % Column 4
\noalign{\hrule height 0.6pt}
% ===== Header =====
&$\Theta$ (rad)&$a^Tb$&$a,b$ $\ldots$&Illustration&\crl
% ===== Rows =====
&$0$&$=\|a\|\|b\|$&are aligned&\epsfbox{dot.3}&\cr
\omit\vrule&$(0,\pi/2)$&$>0$&make an acute angle&\epsfbox{dot.4}&\cr
\omit\vrule&$\pi/2$&$=0$&are orthogonal&\epsfbox{dot.5}&\cr
\omit\vrule&$(\pi/2,\pi)$&$<0$&make an obtuse angle&\epsfbox{dot.6}&\cr
&$\pi$&$=-\|a\|\|b\|$&are anti-aligned&\epsfbox{dot.7}&\cr
\noalign{\hrule height 0.6pt}
}}\smallskip
\itemitem{5.}\emph{Pythagoras}: $a^Tb=\|a\|b\|\cos\Theta$, so $\|a+b\|^2=\|a\|^2
+\|b\|^2$ if $\Theta=\pi/2$ rad\smallskip
\itemitem{6.}\emph{Example} (see Figure): For $v=\left[\matrix{x_1-x_3\cr
y_1-y_3}\right]$ and $w=\left[\matrix{x_2-x_3\cr y_2-y_3}\right]$, we find
$\Theta=\arccos\left({v^Tw\over\|v\|\|w\|}\right)$

\beginsection Figures

\vbox{\offinterlineskip
% ===== Preamble =====
\halign{
\enskip\hfil #\quad\hfil\enskip&  % Column 1
\enskip\hfil #\quad\hfil\enskip&  % Column 2
\enskip\hfil #\quad\hfil\enskip&  % Column 3
\enskip\hfil #\hfil\enskip\cr     % Column 4
% ===== Pictures ====
\epsfbox{dot.1}&\epsfbox{dot.2}&\epsfbox{dot.8}&\epsfbox{dot.9}\cr
% ==== Captions ====
\noalign{\medskip}
C1. $y=\sin x$&C1. $y=\arccos x$&C2. Law of cosines&C6. Example\cr
}}

\bye
